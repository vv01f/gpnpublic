% Created 2016-03-23 Wed 11:52
\documentclass[10pt]{leaflet}
\usepackage[utf8]{inputenc}
\usepackage[T1]{fontenc}

% not needed for releases after 2015, all fixes now in LaTeX kernel
%\usepackage{fixltx2e}

\usepackage{graphicx}
\usepackage{longtable}
\usepackage{float}
\usepackage{wrapfig}
\usepackage{rotating}
\usepackage[normalem]{ulem}
\usepackage{amsmath}
\usepackage{textcomp}

% error message with 2016-05 release, so deactivated
%\usepackage{marvosym}

\usepackage{wasysym}
\usepackage{amssymb}
\usepackage[
	driverfallback=hpdftex,	%hypertex,
]{hyperref}
\tolerance=1000
%\author{Lisa Siebel}

\date{\vspace{-5ex}}
\title{Leitlinien zum Miteinander auf der GPN}
\hypersetup{
  pdfkeywords={},
  pdfsubject={},
  pdfcreator={Emacs 24.5.1 (Org mode 8.2.10)}}
\begin{document}

\maketitle

Hallo und herzlich willkommen auf der Gulaschprogrammiernacht, einer
Veranstaltung mit Technik, leckerem Essen und vielen großartigen
Menschen. Die folgenden Tipps wollen ein Leitfaden zum Umgang mit
diesen sein, um zwischenmenschliche Probleme zu vermeiden und eine
tolle GPN für alle zu ermöglichen.



\section{Allgemeines}
\label{sec-1}
\begin{itemize}
\item Behandle andere Leute möglichst so, wie sie behandelt werden möchten
\item Wenn du dir unsicher bist, frag höflich
\item Wenn jemand gemein ist, versucht, darüber zu reden
\item Wenn es unangenehm wird, oder du Hilfe brauchst, wende dich an
  unsere Vertrauensleute. Wir tragen grüne Badges und sind unter der
  Dectnummer 113  erreichbar. Am Infotresen sind Fotos von uns.
\end{itemize}


\subsection{Erster Eindruck}
\label{sec-1-1}
Wow, ganz schön viele Leute hier...\\
\emph{Jep, letztes Jahr waren es etwa 500 und die GPN wächst weiter}\\\\
Sind das alles Nerds?\\
\emph{Manche wollen sich das ganze auch einfach nur anschauen. Vielleicht
sind ein paar davon ja auch welche von uns und wissen es nur noch
nicht ;-)}\\\\
OK, aber was ist mit\ldots{}?\\
\emph{Zu einigen Gruppen findest du im folgenden weitere
Informationen. Read on.}


\section{Mädchen und Frauen}
\label{sec-2}
Wow, hier sind ja Mädels! Die sind doch bestimmt mit ihrem Freund da?
Und wenn nicht, kann ich sie anbaggern? Ich könnte nämlich ne Freundin
gebrauchen\ldots{}


\emph{Leider ist der Frauenanteil in technischen Berufen ziemlich
  gering. Viele empfinden das Umfeld als unangenehm und bleiben nicht
  lange dabei. Bitte sei nicht der Grund dafür.}

\subsection{Do's}
\label{sec-2-1}
\begin{itemize}
\item Behandle Mädchen und Frauen mit Respekt und Höflichkeit. Viele davon
haben (auch im Chaos) schon miese Erfahrungen gemacht.
\item Frag nach Projekten, Spaces und Interessen, genau wie sonst auch.
\end{itemize}
\subsection{Don'ts}
\label{sec-2-2}
\begin{itemize}
\item Es mag sich seltsam anhören, aber angegraben werden kann echt
  nerven. Vor allem auf Veranstaltungen mit überwiegend Jungs, die
  alle auf diese Idee kommen. Wenn die Situation nicht gerade
  eindeutig auf Flirten hinausläuft, lass es sein. Du versaust ihr den
  Tag damit.
\item Genauso schlimm: Ständig in die Mädchenrolle gedrängt zu
  werden. Die Klassiker sind hier: "Bist du mit deinem Freund hier?
  oder "Was machst denn DU hier?". Es ist absolut mies auf einer
  Chaosveranstaltung ständig die Technikkompetenz abgesprochen zu
  bekommen und es frustriert viele junge Haecksen endlos.
\item Anderesrum stresst es auch, ständig am männlichen Maßstab
  gemessen zu werden und gefühlt nichts gegen eklige, verstörende
  oder sexistische Sprüche  sagen zu dürfen, ohne den Status als cooles
  Mädchen zu verlieren. Anstand und Höflichkeit helfen.
%\item Nicht jede von uns ist Feministin, Expertin für Frauenprobleme
 % oder eine Koryphäe des Sozialen. Es ist ätzend, sich für andere
 % Frauen rechtfertigen oder sie erklären zu müssen, und den modernen
  %Feminismus gegen wikipediaerfahrenen Rethoriker verteidigen zu
  %sollen. Wir sind zum Hacken und für nette Leute hier, lasst den
  %anderen Kram in Ruhe, außer wir zeigen deutliches Interesse daran.
\end{itemize}

\subsection{Wenn du selbst weiblich bist und Stress mit den Jungs hast}
Komm zu uns oder ruf uns an. Wir haben auch Mädels im Team und dulden
keinen Sexismus auf der GPN.


\section{Neurodivergente}
\label{sec-3}
Ein paar Leute hier sind \ldots{} irgendwie seltsam. Sie schauen
z.B. beim Reden weg, schnippen mit den Fingern oder wiegen sich hin
und her und scheinen das lustig zu finden. Irgendwie kommt bei
Gesprächen keine richtige Verbindung auf. Was ist da los?


\emph{Neurodivergenz ist bei weitem noch nicht vollständig
  erforscht. Viele Neurodivergente leiden darunter, dass Neurotypische
  (also Menschen die nicht neurodivergent sint) nicht mit ihnen
  klarkommen oder sie zwanghaft in Stereotypen pressen wollen. Dabei
  würde es auch neurotypischen Menschen helfen, sich mit dieser
  Andersartigkeit auseinander zu setzen und die eigenen Vorstellungen
  zu reflektieren}

\subsection{Do's}
\label{sec-3-1}
\begin{itemize}
\item Auch Neurodivergente sind hier um Spaß am Gerät zu haben und
  nette Leute zu treffen. Behandle sie also so, dass sie sich als Teil
  des Ganzen fühlen können. Ohne Bevormundung.
\item Freundliche Nachfragen werden oft bereitwillig beantwortet. Hier
  braucht es Fingerspitzengefühl.
\end{itemize}

\subsection{Don'ts}
\label{sec-3-2}
\begin{itemize}
\item Niemand mag nur auf einen Aspekt reduziert werden. Lass den
  Leuten auch Raum, eine Person zu sein und reduziere sie nicht auf
  ihre Andersartigkeit. Akzeptiere, wenn sie wütend sind.
\item Bestehe nicht darauf, dass sich alle Neurodivergente wie
  ``typische'' Autisten/Asperger verhalten, auch wenn du solche
  Menschen kennst.
\item Diagnosen sind ganz dünnes Eis. Niemand braucht eine offizielle
  Diagnose um anders zu sein und oftmals wird durch das medizinische
  Establishment sehr viel Schaden angerichtet.
\end{itemize}

\subsection{Wenn du selbst neurodivergent bist und die neurotypischen Probleme machen}
Komm zu uns oder ruf uns an, haben Erfahrung in dem
Bereich und können dir weiterhelfen.

\section{Trans* \& Queer}
\label{sec-5}
Hier sind Leute bei denen ich mir nicht sicher bin, ob sie Männchen
oder Weibchen sind. Und ein paar sehen eindeutig aus, tragen aber
diese Pronomenschilder und da steht was anderes drauf. Und ein paar
davon sind offensichtlich ineinander verliebt. Das ist alles nur Spaß,
oder?

\emph{Geschlecht und Begehren sind sozial konstruiert und werden
  zunehmend in Frage gestellt. Immer mehr Menschen leben als das
  andere, ein drittes , ohne ganz ohne, oder mit wechselndem
  Geschlecht. Andere experimentieren mit diesen
  Ausdrucksformen. Außerdem verlieben sich alle möglichen Leute
  ineinander.}
\subsection{Do's}
\label{sec-5-1}
\begin{itemize}
\item Respektiere Pronomen, Namen und Identitäten. Niemand ist
  verpflichtet dir irgendwas zu beweisen. Frag am besten voher welches
  Prononem passt, statt nur zu schätzen. Auch der Name kann hier ein
  deutlicher Hinweis sein. Wenn du damit absolut nicht klarkommst,
  lass diese Leute in Ruhe und denk in Ruhe drüber nach.
\item Ähnliches gilt für andere Sexualitäten: Wenn du nicht auf
  jemanden stehst, sag das und versuche umgekehrt nicht, Leute zu
  ``bekehren''.
\end{itemize}
\subsection{Don'ts}
\label{sec-5-2}
\begin{itemize}
\item Erschlage Leute nicht mit Stereotypen. Das Fernsehen
  lügt. Youtube auch.
\item Bitte erkläre Leuten nicht, auf welche Toiletten sie gehen oder
  nicht gehen können. Falls es hier zu irgendwelchen Problem kommt
  (wie z.B. Kerlen die solche Vorgaben missbrauchen um Mädchen und Frauen
  zu belästigen), hol uns oder die Orga.
\end{itemize}
\subsection{Wenn du selbst trans* oder queer bist}
Melde dich bei uns, falls irgendwas ist, wir helfen dir weiter. Wenn
du Fragen hast oder einfach nur mit jemandem reden magst, der dich
versteht, wende dich an unsere queere Nonbinäre, Lisa.
\clearpage



\section{Ausländer,  People of Color (PoC), Menschen mit
  Migrationshintergrund}
\label{sec-4}
 Hey, hier sind Leute, die aussehen als ob sie von weit weg
herkämen. Soll ich Ihnen erklären, wie Deutschland so ist?

\emph{Moderne Migrationsbewegungen führen zu vielfältiger Herkunft,
  auch bei Nerds. Diejenigen davon, die auf die GPN kommen, wollen vor
  allem freundliche Menschen treffen und mit spannender Technik
  spielen. Die Kommunikation kann im Einzelfall Probleme machen, aber
  Kosmopolitismus ist zum Glück eine weit verbreitete Eigenschaft
  unter uns Nerds und beide Seiten profitieren davon.}

\subsection{Do's}
\label{sec-4-1}
\begin{itemize}
\item Sei freundlich und hilf Leuten gegebenenfalls beim
  Übersetzen. Im Zweifel sind klare Aussagen besser als
  grammatikalisch elaboriertes Englisch.
\end{itemize}
\subsection{Don'ts}
\label{sec-4-2}
\begin{itemize}
\item Stereotype (auch vermeintlich positive) können sehr schmerzhaft
  sein. Sei dir dessen bewusst und lass Leute damit in Ruhe.
\item Aussehen lässt nicht unbedingt auf Herkunft schließen. Sprich
  Menschen lieber erstmal auf Deutsch an, statt ihnen gleich das
  Gefühl zu geben, hier Fremde zu sein.
\end{itemize}
\subsection{Wenn du selbst einen Migrationshintergrund hast, oder POC bist  und  Hilfe brauchst / If you are a POC or a migrant and want help}
Ruf uns einfach an, wir helfen dir. Just give us a call, we will help
you.




\section{Kinder}
\label{sec-6}
 Wow hier sind ja wirklich kleine Menschen. Haben die sich verlaufen?


\emph{Die GPN zieht Menschen aller Altersklassen an. Außerdem versuchen
diverse Initiativen wie z.B. ``Chaos macht Schule'' auch die Jugend zu
vernünftiger Medienkompetenz zu erziehen.}

\subsection{Do's}
\label{sec-6-1}
\begin{itemize}
\item Du bist älter, sei ein Vorbild.
\item Wenn sich die Kleinen für dein Projekt interessieren, erklär
  es. Benutz dafür altersangemessene Begriffe und wiederhole wichtige
  Informationen. Mach dir nix draus, wenn die Kids trotzdem
  weiterziehen.
\item Versuch dich zu erinnern. Was würdest du dir als Kind in der
  Situation von den Großen wünschen, was davon wäre sinnvoll? Sei
  diese Person.
\end{itemize}


\subsection{Don'ts}
\label{sec-6-2}
\begin{itemize}
\item Bitte sei nicht fies. Wenn du Ruhe brauchst, sag das
  freundlich. Auch Kinder haben Gefühle.
\item Bitte streite dich nicht mit den Kleinen. Ruf lieber uns, wir klären
das.
\end{itemize}
\subsection{Wenn du selbst noch jünger bist und irgendwelche Hilfe brauchst}
Komm zu uns oder ruf uns auf einem Dectphone an.


\section{Diese Regeln sind voll doof und stören meine individuelle
  Freiheit! Was würde passieren, wenn ich aus Trotz folgendes
  subversive Kunstwerk auf die Wände der HFG sprühen würde: Ein in
  leuchtenden Farben gestalter, drei meter hoher, anatomisch
  korrekter\ldots{}}
\label{sec-7}
Make my day, punk. Nein ernsthaft: Wir finden es auch schade, dass wir
überhaupt Leitlinien brauchen, aber leider gibt es zu viele neue Leute
um sich noch auf einen allgemeinen Hackerkonsens zu berufen. Daher
machen wir die Regeln vorher explizit. Wenn du mit uns darüber
streiten magst, besuch uns oder ruf uns an. Am besten fragst du nach
Lisa, die hat auch diesen Flyer verbrochen.
% Emacs 24.5.1 (Org mode 8.2.10)
\end{document}
